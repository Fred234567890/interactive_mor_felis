\section{Ideas}

\subsection{Space Efficient Residual Calculation}
Probably the residual only has to be calculated and considered in coordinates for the adaptive sweep, on which the required post-processing results depend. Accurately, only the field along the beam, since the impedance only depends on it. But this probably doesn't work.

\subsection{Adaptive Residual Calculation}

\subsection{Nested QR MOR}
Port and system matrices can be reduced

\subsection{Multi Mesh MOR}
\begin{itemize}
	\item start on coarse mesh, create MOR
	\item create fine mesh
	\item project MOR basis functions on fine mesh
	\item add basis with the additional adaptive MOR code
\end{itemize}

expected improvements, thoughts: 
\begin{itemize}
	\item Less evaluations of the Model with the fine grid are necessary, since the basis is partially computed a priori
	\item An iterative solver on the fine grid could be started with a initial solution from the MOR
	\item Any solution by the MOR is orthogonal to the new appended basis function. This has to be regarded when using iterative solvers.
	\item Physics like energy conservation are unaffected, since neither the weak formulation nor the Solution space are changed/restricted
\end{itemize}